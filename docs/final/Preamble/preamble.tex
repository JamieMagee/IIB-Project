% ******************************************************************************
% ****************************** Custom Margin *********************************

% Add `custommargin' in the document class options to use this section
% Set {innerside margin / outerside margin / topmargin / bottom margin}  and
% other page dimensions

\ifsetMargin
\else
    \RequirePackage[left=37mm,right=30mm,top=35mm,bottom=30mm]{geometry}
    \setFancyHdr % To apply fancy header after geometry package is loaded
\fi

% *****************************************************************************
% ******************* Fonts (like different typewriter fonts etc.)*************

% Add `customfont' in the document class option to use this section

\ifsetFont
\else
    % Set your custom font here and use `customfont' in options. Leave empty to
    % load computer modern font (default LaTeX font).  

    \RequirePackage{libertine} 
\fi

% *****************************************************************************
% **************************** Custom Packages ********************************


% ************************* Algorithms and Pseudocode **************************

%\usepackage{algpseudocode} 

\usepackage{listings}
\usepackage{color}

\definecolor{mygreen}{rgb}{0,0.6,0}
\definecolor{mygray}{rgb}{0.5,0.5,0.5}
\definecolor{mymauve}{rgb}{0.58,0,0.82}

\lstset { 
  backgroundcolor=\color{white},   % choose the background color; you must add \usepackage{color} or \usepackage{xcolor}
  basicstyle=\footnotesize,        % the size of the fonts that are used for the code
  breakatwhitespace=false,         % sets if automatic breaks should only happen at whitespace
  breaklines=true,                 % sets automatic line breaking
  captionpos=b,                    % sets the caption-position to bottom
  commentstyle=\color{mygreen},    % comment style
  deletekeywords={...},            % if you want to delete keywords from the given language
  escapeinside={\%*}{*)},          % if you want to add LaTeX within your code
  extendedchars=true,              % lets you use non-ASCII characters; for 8-bits encodings only, does not work with UTF-8
  frame=single,                    % adds a frame around the code
  keepspaces=true,                 % keeps spaces in text, useful for keeping indentation of code (possibly needs columns=flexible)
  keywordstyle=\color{blue},       % keyword style
  numbers=left,                    % where to put the line-numbers; possible values are (none, left, right)
  numbersep=5pt,                   % how far the line-numbers are from the code
  numberstyle=\tiny\color{mygray}, % the style that is used for the line-numbers
  rulecolor=\color{black},         % if not set, the frame-color may be changed on line-breaks within not-black text (e.g. comments (green here))
  showspaces=false,                % show spaces everywhere adding particular underscores; it overrides 'showstringspaces'
  showstringspaces=false,          % underline spaces within strings only
  showtabs=false,                  % show tabs within strings adding particular underscores
  stepnumber=1,                    % the step between two line-numbers. If it's 1, each line will be numbered
  stringstyle=\color{mymauve},     % string literal style
  tabsize=2,                       % sets default tabsize to 2 spaces
  title=\lstname                   % show the filename of files included with \lstinputlisting; also try caption instead of title
}


% ********************Captions and Hyperreferencing / URL **********************

% Captions: This makes captions of figures use a boldfaced small font. 
%\RequirePackage[small,bf]{caption}

\RequirePackage[labelsep=space,tableposition=top]{caption} 
\renewcommand{\figurename}{Fig.} %to support older versions of captions.sty

% ************************ Formatting / Footnote *******************************

%\usepackage[perpage]{footmisc} %Range of footnote options 


% ****************************** Line Numbers **********************************

%\RequirePackage{lineno}
%\linenumbers

% *************************** Graphics and figures *****************************

%\usepackage{rotating}
%\usepackage{wrapfig}
%\usepackage{float}
\usepackage{subcaption} %note: subfig must be included after the `caption` package. 
\usepackage{tikz}

% ********************************** Table *************************************

%\usepackage{longtable}
%\usepackage{multicol}
%\usepackage{multirow}
%\usepackage{tabularx}


% ***************************** Math and SI Units ******************************

\usepackage{amsfonts}
\usepackage{amsmath}
\usepackage{amssymb}
%\usepackage{siunitx} % use this package module for SI units


% *********************************** Other ***********************************

\usepackage[showboxes, absolute]{textpos}
\usepackage{setspace}
\usepackage{booktabs}
\usepackage{lscape}
\usepackage{pdfpages}

% *****************************************************************************
% *************************** Bibliography  and References ********************

%\usepackage{cleveref} %Referencing without need to explicitly state fig /table

% Add `custombib' in the document class option to use this section
\ifsetBib % True, Bibliography option is chosen in class options
\else % If custom bibliography style chosen then load bibstyle here

   \RequirePackage[square, numbers, authoryear]{natbib} % CustomBib

% If you would like to use biblatex for your reference management, as opposed to the default `natbibpackage` pass the option `custombib` in the document class. Comment out the previous line to make sure you don't load the natbib package. Uncomment the following lines and specify the location of references.bib file

% \RequirePackage[backend=biber, style=numeric-comp, citestyle=numeric, sorting=nty, natbib=true]{biblatex}
% \bibliography{References/references} %Location of references.bib only for biblatex

\fi


% changes the default name `Bibliography` -> `References'
\renewcommand{\bibname}{References}


% *****************************************************************************
% *************** Changing the Visual Style of Chapter Headings ***************

% Uncomment the section below. Requires titlesec package.

\RequirePackage{titlesec}
\newcommand{\PreContentTitleFormat}{\titleformat{\chapter}[display]{\scshape\Large}
{
%\Large\filleft{\chaptertitlename} \Huge\thechapter  
}
{0ex}{}
[\titlerule]}
%\newcommand{\ContentTitleFormat}{\titleformat{\chapter}[display]{\scshape\huge}
%{\Large\filleft{\chaptertitlename} \Huge\thechapter}{1ex}
%{\titlerule\vspace{1ex}\filright}
%[\vspace{1ex}\titlerule]}
%\newcommand{\PostContentTitleFormat}{\PreContentTitleFormat}
\PreContentTitleFormat


% ******************************************************************************
% ************************* User Defined Commands ******************************
% ******************************************************************************

% *********** To change the name of Table of Contents / LOF and LOT ************

%\renewcommand{\contentsname}{My Table of Contents}
%\renewcommand{\listfigurename}{My List of Figures}
%\renewcommand{\listtablename}{My List of Tables}
%\setlength{\parskip}{\baselineskip}


% ********************** TOC depth and numbering depth *************************

\setcounter{secnumdepth}{2}
\setcounter{tocdepth}{2}

% ******************************* Nomenclature *********************************

% To change the name of the Nomenclature section, uncomment the following line

%\renewcommand{\nomname}{Symbols}


% ********************************* Appendix ***********************************

% The default value of both \appendixtocname and \appendixpagename is `Appendices'. These names can all be changed via: 

%\renewcommand{\appendixtocname}{List of appendices}
%\renewcommand{\appendixname}{Appndx}
\usepackage{appendix}
% ************************** Thesis Abstract *****************************
% Use `abstract' as an option in the document class to print only the titlepage and the abstract.
\begin{abstract}
Optical flow is a critical tool in many experimental methods such as Particle Image Velocimetry for flow analysis and soil mechanics, speckle tracking echocardiography in medical imaging, as well as mechanical testing. During this project Lucas-Kanade, Farnebäck, and SimpleFlow optical flow methods were compared using artificial and real world datasets on the limited hardware of the Raspberry Pi. For dense optical flow Farnebäck was found to have the lowest processing time, SimpleFlow was found to have the highest accuracy, and Lucas-Kanade was found to lie somewhere in between. However, when using Lucas-Kanade as a sparse algorithm it was found to have the lowest processing time. Instructions, aimed at undergraduate level, on how to install the tools used in this project are provided, along with the robust toolset used in testing. Practical applications were also demonstrated by integrating optical flow methods into a tensile testing machine developed as part of the OpenLabTools initiative. Further work investigating performance under arbitrary affine transformations, or projective transformations using the toolset developed as part of this project is possible. Visualisation of vector field showing the curl or the divergence of the vector field may also be a useful tool. 
\end{abstract}

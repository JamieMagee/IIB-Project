\chapter{Conclusions}
\label{sec:conclusions}

\ifpdf
    \graphicspath{{Section5/Figs/Raster/}{Section5/Figs/PDF/}{Section5/Figs/}}
\else
    \graphicspath{{Section5/Figs/Vector/}{Section5/Figs/}}
\fi

It has been demonstrated that executing computer vision methods, specifically optical flow algorithms, on the low power and low cost Raspberry Pi system, it is indeed possible to obtain not only high accuracy, but also high speed results within certain requirements. Performing a comparison between Lucas-Kanade, Farnebäck, and SimpleFlow optical flow algorithms for an artificial dataset comprised of individual affine transformations, and forcing Lucas-Kanade optical flow to be a dense optical flow algorithm, there are obvious choices depending on the application. For applications requiring realtime results Farnebäck optical flow appears to be the best choice. For applications which place accuracy above all else, SimpleFlow optical flow shows the best results. Finally, for applications requiring a compromise between accuracy and processing speed, Lucas-Kanade optical flow is a possible choice.

For real world data the results show a slightly different view, mainly due to Lucas-Kanade optical flow being used as a sparse optical flow method instead of being forced to be a dense method. This is another decision that needs to be made when choosing an optical flow method for an application: whether or not a dense flow field is required, or merely flow vectors for key points in the field. Nevertheless, as a true sparse optical flow method, Lucas-Kanade optical flow performs best in terms of processing time per frame, followed by Farnebäck optical flow then SimpleFlow optical flow.

Another aim of this project was to provide a clear set of instructions aimed at undergraduate level. This was achieved early on in the project as many of the tools, such as OpenCV, were used heavily in this project. The instructions can be seen in code listing~\ref{lst:opencv} in Appendix~\ref{sec:appendix} and will soon be found on the OpenLabTools website. In addition, to show the practical applications of optical flow methods in other areas of engineering, a further aim of the project was to integrate with other projects in the OpenLabTools. One such project was a mechanical testing rig designed and built by Josie Hughes, which allows for tensile testing of materials. Using optical flow methods the strain can be measured, and there is recent research which allows the Poisson ratio to be calculated from flow field data from optical flow methods~\cite{chen2013new}. While this project does not explore these avenues, there is certainly scope for doing so using the robust tool set used in this project.

There is scope for extension in several areas of this project. Within the comparison only individual affine transformations were explored, whereas combinations of affine transformations are not. Single affine transformations are less likely to occur in real world data, than arbitrary affine transformations, therefore it may be worth investigating. The tools used to analyse the artificial dataset was written to support, not only single affine transformations, or even arbitrary affine transformations, but projective transformations too. Therefore it should be relatively simple to pursue this area of research. Better visualisation of the vector field is also another possible extension. Currently the vector field can be visualised using vector arrows, or as a colour mapping to HSV colour space, however showing the curl or the divergence of the vector field may also be a useful tool. If the vector field represents the flow velocity of a moving fluid, then the curl is the circulation density of the fluid and the divergence measures the magnitude of a vector field's source or sink at a given point.